\chapter{February}

\section{Feb 1st}\index{Feb 1st}
\subsection{Standard deviation}
\textbf{Standard deviation} $\sigma$ is defined as the square root of \textbf{variance}. That is,
\begin{align}
	\sigma(X) &= \sqrt{var(X)}
\end{align}
The definition for variance:
\begin{align}
var(X) = E[(X - \mu_X)^2]
\end{align}
The definition for \textbf{covariance}:
\begin{align}
cov(X,Y) = E[(X - \mu_X)(Y- \mu_Y)]
\end{align}
A property:
\begin{align}
	cov(X,X) = var(X) = \sigma(X)^2 
\end{align}
Definition for \textbf{correlation}:
\begin{align}
	\begin{split}
	\rho(X,Y) & = corr(X, Y) 		\\
			  & = \frac{cov(X,Y)}{\sigma_X \sigma_Y}
	\end{split}
\end{align}

\section{Feb 2nd}\index{Feb 2nd}

\subsection{Determinant}
Today when I try to compute the determinant of the covariance matrix in the multivariate Gaussian, I come across the problem of overflow. In fact, I only
need to know the logarithm of the determinant. Therefore, I apply the following solution:
\begin{align}
	\begin{split}
	\log(\det A) &= \log(\Pi_{i=1}^{N} \lambda_i)  \\
				 & = \sum_{i=1}^{N} \log(\lambda_i)
	\end{split}
\end{align}
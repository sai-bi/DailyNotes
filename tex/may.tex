\chapter{May}

\section{Build sparse matrix with Eigen efficiently} \index{Build sparse matrix with Eigen efficiently}

When I build a sparse matrix with Eigen, I come across a problem that it is quite slow. The problem is that
when you \textit{insert} elements one by one, each time Eigen needs to reallocate space for the matrix and copy
the data, which is reason that it can be slow. 

The solution is to first record the row, column index as well as value of non-zero entries with \textit{triplets},
and then build the sparse matrix with the function \textit{setFromTriplets}.

\chapter{June}
\section{Transform matrix and projective matrix} \index{Transform matrix and projective matrix}
To transform a vertex from world space to camera space, we need to construct the rigid transform matrix. Given a right
hand system, we need \textbf{eye} $\in \mathbb{R}^3$ position, \textbf{lookAt} $\in \mathbb{R}^3$ position as well as \textbf{up} $\in \mathbb{R}^3$ direction to determine the matrix.
The matrix is in the following format:
 \myequ{
V & =       \begin{pmatrix}
           s.x & u.x & -f.x & 0.0 \\
           s.y & u.y & -f.y & 0.0 \\
           s.z & u.z & -f.z & 0.0 \\
           - (s \cdot eye) & -(u \cdot eye) & -(f \cdot eye) & 1.0  
       \end{pmatrix} \\
  where\  & f = normalize(lookAt - eye) \\
  & s = normalize(f \times up) \\
  & u = s \times f 
}

\begin{remark}
    Note this matrix is constructed for right-hand coordinate system.  
\end{remark}

In addition, to transform a point from camera space to image space, we need to construct the perspective projection matrix. 
In this case, we need field of view in $x$ and $y$ axis, namely \textbf{fovX} and \textbf{fovY}, \textbf{nearClip} and \textbf{farClip}. The matrix is build as:
 
 \myequ{
     P & =       \begin{pmatrix}
        \frac{1}{\text{tan}(\frac{fovX}{2})} & 0 & 0 & 0 \\
        0 & \frac{1}{\text{tan}(\frac{fovY}{2})} & 0 & 0 \\
        0 & 0 & -\frac{far + near}{far - near} 
        &  -\frac{2 * far * near}{far - near} \\
        0 & 0 & -1 & 0 
 \end{pmatrix} 
}










